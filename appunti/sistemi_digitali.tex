\documentclass[a4paper,oneside]{book}
\linespread{1.5}
\usepackage[left=2.5cm, right=2.5cm, top=2.5cm, bottom=2.5cm]{geometry}
\usepackage[utf8]{inputenc}
\usepackage[italian]{babel}
\usepackage{amsfonts}
\usepackage{amsmath}
\usepackage{amsthm}
\usepackage{amssymb,amsmath,color}
\usepackage{cite}
\usepackage{graphicx}
\usepackage{float}
\usepackage{algorithm}
\usepackage{algpseudocode}
\usepackage{algorithmicx}
\usepackage{pdfpages}

\begin{document}
    \title{
        Appunti di Sistemi Digitali D \\
        \normalsize Sas
        }
	\author{Filippo Lenzi e Guglielmo Palaferri}

    \maketitle

	\tableofcontents
    \thispagestyle{empty}

    \chapter{Intro}

    Versione attuale del doc: trascrizione videolezioni circa, da
    mettere meglio a posto in seguito

    \section{22 settembre 2021}

    Scopo del corso: introduzione a sistemi embedded

    Diverse famiglie: risorse limitate con meno consumo, o viceversa

    Metodologie di sviluppo per questi dispositivi

    Più vicino a hardware con linguaggi di basso o alto livello, 
    o anche smartphone con linguaggi di più alto livello

    Si userà Python, linguaggio ad oggi più utilizzato (doubt)

    In contesto anche applicativo: usato in applicazioni reali

    Modulo 1: studio dispositivi (come FPGA) su cui si possono usare 
    linguaggi di basso livello, vengono usati nel corso framework di 
    più alto livello come HLS, che usa variante di C/C++, non Verilog e 
    VHDL che sarebbero molto più vicini all'hardware

        Inoltre si definiscono modalità di calcolo / progetto del sistema
        stesso in modulo 1

    Modulo 2: come usare architetture esistenti, e anche board con bridge 
    in Python per manipolare hardware sottostante

    Python è interpretato ed è meno performante di C, avendo più overhead

    Info esame

    esame MASSIMO 2 VOLTE

    \subsubsection{Sistemi Embedded}

    Sistemi digitali includono tutti i dispositivi a ogni livello di consumo; i
    \textbf{sistemi embedded} sono compatti e a basso consumo energetico; quindi, 
    la complessità delle funzioni da svolgere è un fattore limitante, dovendo spesso
    gestire una batteria limitata, e evitare altri problemi come surriscaldamento in 
    base al caso d'uso.

    Board: scheda con diversi accessori, con un piccolo dispositivo centrale. 
    Più specializzata, architettura progettata ad hoc per la funzione. Hanno 
    comunque delle librerie per interfacciare con accessori in usecase comuni 
    (fotocamere, usb, ecc.).

    Alcuni esempi di dispositivi:

    \begin{itemize}
        \item Robot di pulizia casalinga, con software di navigazione in base
            a sensori.
        \item Occhiali per persone ipovedenti, con fotocamera che elabora informazioni 
            per fornire info su semafori, testo e altro in forma audio, che necessita 
            di lunga autonomia e limitazione del calore.
        \item Droni contengono sistema embedded per ricevere il controllo remoto e tradurlo
            in movimento delle eliche.
        \item Sblocco del telefono con fotocamera: gestione batteria, fotocamera deve rimanere in 
            "semi-idle" per poter rilevare volto, senza consumare troppa batteria.
    \end{itemize}

    Discorso di bilanciamento tra ottimizzare tempo performance mettendo cose in memoria, 
    consumando però più memoria

    Computer vision: elaborazione immagini. La mole di dati da elaborare può essere 
            molto grande, essendo matrici. Spesso basato su machine learning. 
            Esiste hardware dedicato al processamento di immagini.

    Elaborazione di stream video: mole di dati dipende da framerate oltre che da risoluzione

    Due strategie viste nel corso per elaborare video/immagini:

    \begin{itemize}
        \item Modulo 1: Board con ARM + FPGA, paradigma di programmazione classico di basso livello,
            anche con astrazione via Python in certi casi. Architettura progettata apposta per 
            il problema. Non general purpose, programmazione più complessa.
        \item Modulo 2: Architettura hardware già progettata e immutabile (smartphone, GPU), con programmazione
            e tool di più alto livello, più simili a PC desktop. Controllo sull'hardware limitato 
            da API sul dispositivo, e quindi limite alla complessità dei programmi dettato da esse.
    \end{itemize}

    \chapter{Modulo 1}
    
    \chapter{Modulo 2}
    

\end{document}