\documentclass[a4paper,oneside]{book}
\linespread{1.5}
\usepackage[left=2.5cm, right=2.5cm, top=2.5cm, bottom=2.5cm]{geometry}
\usepackage[utf8]{inputenc}
\usepackage[italian]{babel}
\usepackage{amsfonts}
\usepackage{amsmath}
\usepackage{amsthm}
\usepackage{amssymb,amsmath,color}
\usepackage{cite}
\usepackage{graphicx}
\usepackage{float}
\usepackage{algorithm}
\usepackage{algpseudocode}
\usepackage{algorithmicx}
\usepackage{pdfpages}

\begin{document}
    \title{
        Appunti di Sistemi Digitali D \\
        \normalsize Sas
        }
	\author{Filippo Lenzi e Guglielmo Palaferri}

    \maketitle

	\tableofcontents
    \thispagestyle{empty}

    \chapter{Intro}

    Versione attuale del doc: trascrizione videolezioni circa, da
    mettere meglio a posto in seguito

    \section{22 settembre 2021}

    Scopo del corso: introduzione a sistemi embedded

    Diverse famiglie: risorse limitate con meno consumo, o viceversa

    Metodologie di sviluppo per questi dispositivi

    Più vicino a hardware con linguaggi di basso o alto livello, 
    o anche smartphone con linguaggi di più alto livello

    Si userà Python, linguaggio ad oggi più utilizzato (doubt)

    In contesto anche applicativo: usato in applicazioni reali

    Modulo 1: studio dispositivi (come FPGA) su cui si possono usare 
    linguaggi di basso livello, vengono usati nel corso framework di 
    più alto livello come HLS, che usa variante di C/C++, non Verilog e 
    VHDL che sarebbero molto più vicini all'hardware

        Inoltre si definiscono modalità di calcolo / progetto del sistema
        stesso in modulo 1

    Modulo 2: come usare architetture esistenti, e anche board con bridge 
    in Python per manipolare hardware sottostante

    Python è interpretato ed è meno performante di C, avendo più overhead

    Info esame

    esame MASSIMO 2 VOLTE

    \subsubsection{Sistemi Embedded}

    Sistemi digitali includono tutti i dispositivi a ogni livello di consumo; i
    \textbf{sistemi embedded} sono compatti e a basso consumo energetico; quindi, 
    la complessità delle funzioni da svolgere è un fattore limitante, dovendo spesso
    gestire una batteria limitata, e evitare altri problemi come surriscaldamento in 
    base al caso d'uso.

    Board: scheda con diversi accessori, con un piccolo dispositivo centrale. 
    Più specializzata, architettura progettata ad hoc per la funzione. Hanno 
    comunque delle librerie per interfacciare con accessori in usecase comuni 
    (fotocamere, usb, ecc.).

    Alcuni esempi di dispositivi:

    \begin{itemize}
        \item Robot di pulizia casalinga, con software di navigazione in base
            a sensori.
        \item Occhiali per persone ipovedenti, con fotocamera che elabora informazioni 
            per fornire info su semafori, testo e altro in forma audio, che necessita 
            di lunga autonomia e limitazione del calore.
        \item Droni contengono sistema embedded per ricevere il controllo remoto e tradurlo
            in movimento delle eliche.
        \item Sblocco del telefono con fotocamera: gestione batteria, fotocamera deve rimanere in 
            "semi-idle" per poter rilevare volto, senza consumare troppa batteria.
    \end{itemize}

    Discorso di bilanciamento tra ottimizzare tempo performance mettendo cose in memoria, 
    consumando però più memoria

    Computer vision: elaborazione immagini. La mole di dati da elaborare può essere 
            molto grande, essendo matrici. Spesso basato su machine learning. 
            Esiste hardware dedicato al processamento di immagini.

    Elaborazione di stream video: mole di dati dipende da framerate oltre che da risoluzione

    Due strategie viste nel corso per elaborare video/immagini:

    \begin{itemize}
        \item Modulo 1: Board con ARM + FPGA, paradigma di programmazione classico di basso livello,
            anche con astrazione via Python in certi casi. Architettura progettata apposta per 
            il problema. Non general purpose, programmazione più complessa.
        \item Modulo 2: Architettura hardware già progettata e immutabile (smartphone, GPU), con programmazione
            e tool di più alto livello, più simili a PC desktop. Controllo sull'hardware limitato 
            da API sul dispositivo, e quindi limite alla complessità dei programmi dettato da esse.
    \end{itemize}

    \section{24 settembre 2021}

    GPU è progettata per velocizzare operazioni grafiche di processori esistenti; 
    ARM e FPGA costituiscono un sistema integrato con una acceleratore progettato 
    apposta per essa, e con la quale si scambia informazioni su una linea molto rapida, 
    consumando meno energia di una GPU general purpose.

    Alcune problematiche attuali

    Un telefono di fascia medio-alta spesso ha 3-5 fotocamere: questa visuale stereo può
    essere usata per ottenere diverse informazioni, per esempio una mappa della distanza 
    di ogni pixel dalla telecamera, o il riconoscimento semantico di diversi oggetti.
    Questi problemi hanno un consumo di risorse anche alto, e non necessariamente una rete 
    pensata per un contesto funzionerà bene in altri.

    L'analisi della profondità ha molti usi, anche semplicemente il calcolo dell'occlusione 
    (oggetti reale davanti che nasconde oggetto virtuale dietro) per realtà aumentata.

    Molti casi comuni sono pensati per essere eseguiti su GPU con consumi molto alti e
    quindi non adatti a sistemi embedded. Lo scopo è avere reti utilizzabili su sistemi 
    embedded con prestazioni comparabili (sia framerate/velocità, che accuratezza e qualità).

    Esistono framework di ottimizzazione di sistemi per dispositivi fatti da diversi 
    produttori (come Apple e Intel): per esempio, un'ottimizzazione comune è evitare 
    float e usare interi.

    Esempio: adattamento di rete pensata per GPU a FPGA e telefono. Lo scopo è quindi
    usare questi framework per ottimizzare reti (soprattutto di Deep Learning) a 
    dispositivi a basso consumo; per algoritmi fatti dal programmatore, l'ottimizzazione 
    verte più su di lui.

    Un sistema basato su deep learning richiede calcoli anche pesanti, e ha un problema in 
    base alla disponibilità dei dati di training: se allenato su un contesto specifico, non 
    sarà in grado di operare su altri contesti. Si cercano di creare in quel caso sistemi che 
    apprendono dai nuovi dati, aggiustando parametri della rete in base a nuovi ambienti.

    Elaborazione stream video: milioni di pixel su schermo, ognuno elaborabile in diverso modo, 
    e quindi originalmente GPU nate per questo, ma si presta bene anche per reti neurali.

    Processori ARM hanno minori consumi, inizialmente usati per telefoni, poi introdotti 
    a PC date le prestazioni comparabili.

    Oggi la frequenza di clock di una CPU è meno importante, visto che comporta consumi più 
    elevati. Spesso la frequenza è modulata in base al carico ad oggi.

    Alcuni esempi di architettura a consumo ridotto:

    \begin{itemize}
        \item Movidius Intel: acceleratore per immagini, una sorta di micro-GPU, presente su telefoni 
        e anche in formato USB.
    
        \item JevoIs Smart Machine Vision: anche esso molto piccolo, contiene processore ARM, con 
        telecamera integrata, lunga autonomia. 
    
        \item Jetson Nvidia: sistema con piccola GPU.
    
        \item TPU Google: accelera calcoli di reti machine learning, pure qua ARM. Usato da loro per calcoli 
        correlati invece dei server general purpose, per cose come il riconoscimento vocale ecc.
        TPU sta per Tensor Processing Unit, è una architettura custom per massimizzare la performance 
        nei datacenter.
    
        \item Brainwave di Microsoft: non architettura custom completamente, ma utilizzando board con FPGA 
        con design originale della rete. Usato non per training, ma per inferenza di reti.
    \end{itemize}

    Gli esempi precedenti non sono programmabili a livello di architettura, solo nel software.
    Negli ultimi anni diversi produttori hanno reso disponibili dei chip, basati su una serie 
    di processori (solitamente ARM) e un FPGA, che permettono di definire qualunque architettura. 
    La CPU permette di fare qualunque task, anche se non al meglio, mentre la FPGA è totalmente 
    progammabile per ottenere uno scopo specifico, accelerando i calcoli. 
    Addirittura, è possibile definire delle architetture in modo automatico a partire da una descrizione 
    fornita in C o C++. Lato negativo è quindi richiedere più tempo per definire l'architettura, 
    rispetto ai sistemi già configurati dei vari produttori.

    RISC-V: serie di istruzioni open-source per CPU RISC, supportato da molte aziende (a parte ARM).
    Set di istruzioni libero ed estendibile, con un comitato affinchè l'ISA non cambi.
    ARM come azienda vende progetti, non hardware, anche a aziende rivali. Il suo possibile acquisto 
    da parte di NVIDIA sta aumentando l'interesse in RISC-V.

    Altri sistemi embedded con architettura non programmabile, linguaggio spesso Python: raspberry, 
    jetson Nvidia, smartphone, ecc. Con Python, il framework di machine learning più comune è 
    Tensorflow di Google.

    Sistemi programmabili: un esempio trattato nel corso architettura ZYNQ prodotta da Xilinx, appunto 
    con dual core ARM e un FPGA, collegati tramite bus ad alta efficienza AXI; possiede inoltra memoria 
    condivisa tra CPU e FPGA. Su slide presenti le caratteristiche tecniche (PS = Processing System) della 
    CPU. Usa sistema operativo Linux di base, nel corso si usa OS "bare metal" cioè senza protezioni, molto 
    basilare. All'avvio una delle CPU decide quale bitstream usare per configurare FPGA. 
    Altra architettura più recente Ultrascale, con prestazioni elevate e una simile idea generale di 
    progettazione. Infine, la PYNQ, simile a ZYNQ; si crea l'architettura con linguaggi di alto livello (C) 
    come con ZYNQ, e su di essa si può scrivere codice con Python. Queste board sono simili a quelle basate su 
    Xilinx-700 (ZedBoard), ma si può usare un linguaggio di alto livello.

    È possibile simulare queste board su software, nel caso non si posseggano.

    Su slide un esempio di progetto su ZYNQ.

    In conclusione: Modulo 1 basato su architetture custom, Modulo 2 strumenti di ottimizzazione per sfruttare 
    architetture esistenti.

    \chapter{Modulo 1}
    
    \chapter{Modulo 2}

    \section{06 dicembre 2021}

    \textbf{Introduzione al Deep Learning}

    Vedremo una panoramica sul mondo del deep learning,
    in particolare dal punto di vista dello sviluppatore,
    senza soffermarsi troppo sui dettagli teorici.

    Due framework principali: Tensorflow e PyTorch.

    Due strade possibili:
        \begin{itemize}
        \item utilizzare una rete neurale
        già addestrata, integrandola all'interno del proprio
        progetto software, oppure
        \item addestrare autonomamente una rete neurale per poi
        andarla ad utilizzare 
        \end{itemize}

    Per deep learning si intende una famiglia di metodi
    basati su reti neurali che tramite l'apprendimento da
    insiemi di dati etichettati possono consentire di risolvere
    problemi generalizzati su un qualsiasi insieme di dati.
    
    Teorizzati inizialmente negli anni '80, hanno trovato
    ampia diffusione solo in tempi recenti, quando è stato
    possibile soddisfare due requisiti fondamentali per l'utilizzo
    di reti neurali:
    \begin{itemize}
        \item Elevata potenza di calcolo
        \item Grandi quantità di dati per l'addestramento 
    \end{itemize}

    Le reti neurali convoluzionali (CNNs) sono tra i framework più popolari, 
    consentono di risolvere problemi di classificazione e regressione nell'ambito dell'elaborazione delle immagini.
    Due fasi: Training e Testing

    Una rete neurale è definita da una sequenza di moduli
    (livelli), ciascuno caratterizzato da un insieme di pesi (ognuno associato ad un neurone). 
    I pesi definiscono il comportamento di ogni modulo e di conseguenza della rete nella sua interezza. 
    Durante la fase di Training, la rete processa il dato fornito in input (nel nostro caso un'immagine)
    e predice un risultato. Tale risultato viene confrontato con l'etichetta reale assegnata all'immagine. 
    Misurando la differenza tra il valore predetto dalla rete e l'etichetta possiamo quantificare l'errore 
    ed ottimizzare la rete per minimizzarlo, in particolare aggiustando i valori dei vari pesi della rete.

    La sequenza di moduli che definiscono la rete può essere percorsa
    in avanti (\textbf{forward pass}) per ottenere l'output a partire dall'input,
    oppure all'indietro (\textbf{backward pass}) per ottenere l'espressione
    dell'uscita in funzione dei pesi della rete (poiché l'uscita è funzione composta
    delle uscite dei singoli moduli).

    A livello matematico, la differenza tra il risultato della rete e l'etichetta viene modellata tramite una funzione L 
    (\textit{loss function}), calcolata come la differenza tra l'output (funzione degli N pesi della rete) e l'etichetta. 
    La differenza viene minimizzata muovendo i pesi in direzione opposta rispetto al gradiente della funzione L.

    *espressione matematica del peso $\theta_j$*

    Framework per il deep learning quali Tensorflow e PyTorch gestiscono automaticamente 
    l'aggiornamento dei pesi nella fase di training della rete.

    *esempi di problemi risolvibili con una CNN: object detection, instance segmentation, depth estimation ecc.*

    Vedremo ora il funzionamento dei due framework citati, mostrando infine
    come sia possibile eseguire il porting su dispositivi mobili (in particolare Android).

    \textbf{Tensorflow}

    Sviluppato da Google, esistono due versioni principali:
    1.x e 2.x.
    In Tensorflow 1.x, lo sviluppo del codice era diviso in 
    due fasi: costruzione del grafo (trainig) ed esecuzione (testing).
    In Tensorflow 2.x al contrario, non è prevista la separazione tra le due fasi.

    Tensorflow si basa sul concetto di \textbf{tensore}, ovvero
    un array multidimensionale: array (1-D), matrici (2-D), array di matrici (3-D) ecc.
    Possiamo pensare ad un'immagine come ad un tensore 3-D (altezza, larghezza e
    canale RGB).

    La struttura della rete viene rappresentata tramite un \textbf{grafo} unidirezionale,
    il quale rappresenta l'esatta successione dei layer. Questo consente in modo efficiente
    sia di percorrere il grafo per ottenere l'output in fase di forward pass,
    sia di recuperare le dipendenze tra i diversi layer in fase di backward pass.
    In questo modo si modella la rete tramite un'astrazione ad alto livello.

        


\end{document}