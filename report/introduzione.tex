\chapter*{Introduzione}

Questo progetto è ispirato ad un lavoro del 2018 intitolato 
\textit{``Super SloMo: High Quality Estimation of Multiple Intermediate Frames for Video Interpolation''}
(disponibile a \href{http://jianghz.me/projects/superslomo/}{questo link}).
Tale studio illustra l'uso di reti neurali convoluzionali (CNN) per svolgere l'\textit{interpolazione dei frame} in un video, ovvero la generazione
di (uno o più) fotogrammi intermedi a partire da due frame consecutivi in modo da aumentare il frame rate.

Lo scopo del progetto era quello di realizzare un'applicazione Android
che implementasse un sistema di slow motion "artificiale" tramite le tecniche esposte nell'articolo sopra citato.
In particolare, è stata utilizzata un'implementazione del modello in Pytorch disponibile su GitHub, la quale è stata adattata per l'utilizzo su un dispositivo Android.