\chapter{Sviluppo Android}

\section{Basi Android}
Gu


\section{Conversione video}

L'applicazione necessita di un metodo per convertire i video in immagini contenenti i
fotogrammi, e viceversa convertire le immagini in video una volta finita l'elaborazione.
Si è scelto di usare Ffmpeg, la più diffusa software suite di elaborazione media.

\subsubsection*{FFmpeg}

FFmpeg è il framework multimedia più diffuso sul mercato. Offre la possibilità di convertire, 
registrare, e riprodurre la maggior parte dei formati video e audio. In particolare, 
\textbf{FFmpeg} vero e proprio è la suite di conversione, encoding, e decoding di file
multimediali, nella forma di un programma da linea di comando estremamente flessibile.
\textbf{FFplay} è un media player basato sulle stesse librerie, e \textbf{FFprobe} è un
programma per l'analisi di stream multimediali. Nella maggior parte dei casi, quando si
parla di FFmpeg si intende il programma di conversione di file multimediali.

FFmpeg è gratuito e open source, ed è usato come backend di molte applicazioni di elaborazione
media, sia mobile che desktop. Supporta la maggior parte dei sistemi operativi, tramite librerie
ufficiali o create da utenti terzi, e in particolare, per lo scopo di questo progetto, anche
Android. 

Diverse librerie costituiscono la base di FFmpeg. Le più importanti per la funzione di
conversione sono \emph{libavcodec}, contenente decoder e encoder per molti codec audio e video, 
e \emph{libavformat}, contenente \emph{muxer} e \emph{demuxer} (multiplexer e demultiplexer, 
per l'unione o separazione di canali multimedia) di vari formati media.

L'uso del programma da linea di comando è il seguente: vengono specificati uno o più file di 
input, uno o più file di output, e eventuali filtri e parametri opzionali. Per esempio:

\begin{verbatim}
-- Conversione tra formati audio
ffmpeg -i audioin.wav audioout.mp3
-- Estrazione di fotogrammi come immagini PNG
ffmpeg -i videoin.mp4 frames/%06d.png
-- Uso di filtri video per scalare e ruotare l'immagine, e altri parametri
ffmpeg -i videoin.mp4 -vsync 0 -vf "transpose=1,scale=-1:180" frames/%06d.png
\end{verbatim}

In questo progetto, FFmpeg è stato usato per estrarre i fotogrammi come immagini PNG dal video
di input, e unire i fotogrammi elaborati da Super SlowMo in un video con gli stessi fotogrammi
per secondo dell'originale, e durata quindi raddoppiata.

\subsubsection*{FFmpegKit}

Per poter usare FFmpeg su Android, è stata usata la libreria FFmpegKit, che include strumenti
per eseguire FFmpeg su Android, insieme ad altre piattaforme: iOS, macOS, tvOS, Flutter, e 
React Native. Offre diversi script per fare build delle librerie di FFmpeg, e una libreria 
wrapper per lanciare i comandi di FFmpeg e FFprobe all'interno delle applicazioni. 

Per il contesto di questo progetto, FFmpegKit è stata usata per lanciare i comandi di FFmpeg
all'interno dell'applicazione. Per questo scopo, offre una classe \texttt{FFmpegKit}, che può
essere usata in questo modo, usando come esempio uno dei comandi mostrati sopra:

\begin{lstlisting}
FFmpegSession session = FFmpegKit.execute("-i videoin.mp4 frames/%06d.png");
if (ReturnCode.isSuccess(session.getReturnCode())) {
    Log.i("App", "Extraction success");
} else if (ReturnCode.isCancel(session.getReturnCode())) {
    Log.i("App", "Extraction canceled");
} else {
    Log.d("App", String.format("Extraction failed with state %s and rc %s.%s", session.getState(), session.getReturnCode(), session.getFailStackTrace()));
}
\end{lstlisting}

L'istanza di \texttt{FFmpegSession} offre diversi metodi per ottenere informazioni sulla sessione
in corso, specie nel caso in cui sia creata lanciando il comando in modo asincrono. Per esempio,
questo codice permette di cancellare l'operazione dopo 5 secondi se non ancora conclusa:

\begin{lstlisting}
FFmpegSession session = FFmpegKit.executeAsync("-i videoin.mp4 -vsync 0 -vf \"transpose=1,scale=-1:180\" frames/%06d.png"),
    new FFmpegSessionCompleteCallback() {
        @Override
        public void apply(FFmpegSession session) {
            ReturnCode returnCode = session.getReturnCode();
            Log.i("App", String.format("Extraction success with return code %s", returnCode));
        }
    });
Thread.sleep(5000);
if (session.getState() != SessionState.COMPLETED)
    session.cancel();
\end{lstlisting}

La libreria viene inclusa all'interno del progetto semplicemente aggiungendola a 
\emph{build.gradle}:

\begin{verbatim}
dependencies {
    [...]
    implementation 'com.arthenica:ffmpeg-kit-full:4.5.1-1'
}
\end{verbatim}

\subsubsection*{ConvertVideo}

Per semplificare l'uso di FFmpegKit nel contesto dell'applicazione, e ridurre il rischio
di errori, comandi mal formati, o parametri dimenticati, è stata realizzata un'ulteriore 
classe \texttt{ConvertVideo}, che fa da wrapper a FFmpegKit per aggiungere parametri tramite 
metodi specifici, senza costruire manualmente la stringa del comando.

Un esempio di uso per estrarre fotogrammi, scalarli a 180p, e ruotarli (supponendo quindi un 
video verticale di partenza), è il seguente:

\begin{lstlisting}
convertVideo.rotateClockwise();
convertVideo.setResize(320, 180);
boolean convertSuccess = convertVideo.extractFrames(
    selectedFile,
    extractedFramesDir.getAbsolutePath()
);
\end{lstlisting}

che è equivalente a 

\begin{verbatim}
ffmpeg -i $selectedFile -vsync 0 -vf "transpose=1,scale=320:180" 
    $extractedFramesDir/%06d.png
\end{verbatim}

Questo assicura di evitare errori nell'impostazione dei parametri a livello di compilatore, 
che altrimenti non sarebbero notati fino al tempo di esecuzione data la loro collocazione dentro
ad una stringa.

La classe offre un metodo per estrarre fotogrammi da un video:
\begin{lstlisting}
extractFrames(inVideoPath, outFramesDir)
\end{lstlisting}

e un metodo per unire dei fotogrammi in un video:
\begin{lstlisting}
createVideo(inFramesDir, outVideoPath, fps)
\end{lstlisting}

Offre inoltre alcuni metodi per impostare il ridimensionamento dell'immagine, e la trasposizione,
cioè la rotazione, oltre che metodi per resettare alle impostazioni originali:

\begin{lstlisting}
setResize(int width, int height)
resetResize()
rotateClockwise()
rotateCounterClockwise()
resetRotate()
\end{lstlisting}

La classe viene quindi usata prima e dopo l'elaborazione tramite la classe \texttt{SlowMo}, 
per estrarre i fotogrammi dal video di input e creare il video di output a partire dai
fotogrammi elaborati.

\section{Struttura applicazione}
Gu


FONTI (METTERE IN BIBLIOGRAFIA)

https://ffmpeg.org/
https://github.com/tanersener/ffmpeg-kit